% stuff to be used whe compiling only one chapter -- e.e for revision and committee approval

\documentclass[leqno]{article}

%% Compile this with
%% R> library(pgfSweave)
%% R> pgfSweave

%\usepackage[nogin]{Sweave} %way to understand R code
\usepackage{pgf} %look it up
\usepackage{tikz} % allows R code or other stuff in the figure labels
\usepackage{rotating}
\usepackage{color}
\definecolor{greytext}{gray}{0.5}
\usepackage[left=1.2in,right=1.2in,top=1.2in,bottom=1.2in]{geometry} % sets the margins
\usepackage{fancyhdr}
\usepackage{setspace}
\usepackage{indentfirst}
\usepackage{titlesec}
\usepackage{natbib}
\usepackage{checkend}   % better error messages on left-open environments
\usepackage{graphicx}   % for incorporating external images

\newcommand{\lang}{\textsf} %
\newcommand{\code}{\texttt}
\newcommand{\pkg}{\texttt}
\newcommand{\ques}[1]{{\bf\large#1}}
\newcommand{\eb}{\\ \nonumber} 

\doublespacing

\usepackage{times,mathptmx,courier}
\usepackage[scaled=1]{helvet}
\usepackage{lineno}
\usepackage{listings}
\lstset{basicstyle=\sffamily\scriptsize,showstringspaces=false,fontadjust}

\usepackage{fixme}%you probably wan't this 
%% Math or theory people may want to include the handy AMS macros
\usepackage{amssymb}
\usepackage{amsmath}
\usepackage{amsfonts}
\usepackage{float}

%% The pifont package provides access to the elements in the dingbat font.   
%% Use \ding{##} for a particular dingbat (see p7 of psnfss2e.pdf)
%%   Useful:
%%     51,52 different forms of a checkmark
%%     54,55,56 different forms of a cross (saltyre)
%%     172-181 are 1-10 in open circle (serif)
%%     182-191 are 1-10 black circle (serif)
%%     192-201 are 1-10 in open circle (sans serif)
%%     202-211 are 1-10 in black circle (sans serif)
%% \begin{dinglist}{##}\item... or dingautolist (which auto-increments)
%% to create a bullet list with the provided character.
\usepackage{pifont}
\usepackage[printonlyused,nohyperlinks]{acronym}

\usepackage{comment}



\titleformat*{\section}{\singlespacing\raggedright\bfseries\Large}
\titleformat*{\subsection}{\singlespacing\raggedright\bfseries\large}
\titleformat*{\subsubsection}{\singlespacing\raggedright\bfseries}
\titleformat*{\paragraph}{\singlespacing\raggedright\itshape}

%% The caption package provides support for varying how table and
%% figure captions are typeset.
\usepackage[format=hang,indention=-1cm,labelfont={bf},margin=1em]{caption}

%% url: for typesetting URLs and smart(er) hyphenation.
%% \url{http://...} 
\usepackage{url}
\urlstyle{sf}   % typeset urls in sans-serif
%\graphicspath{{/home/daft/Dropbox/figures/}{/home/daft/Dropbox/figures/cpue_figures/}} % set the path to figures folder
%%%%%%%%%%%%%%%%%%%%%%%%%%%%%%%%%%%%%%%%%%%%%%%%%%%%%%%%%%%%%%%%%%%%%%
%%%%%%%%%%%%%%%%%%%%%%%%%%%%%%%%%%%%%%%%%%%%%%%%%%%%%%%%%%%%%%%%%%%%%%
%%
%% Possibly useful packages: you may need to explicitly install
%% these from CTAN if they aren't part of your distribution;
%% teTeX seems to ship with a smaller base than MikTeX and MacTeX.
%%
%\usepackage{pdfpages}  % insert pages from other PDF files
%\usepackage{longtable} % provide tables spanning multiple pages
%\usepackage{chngpage}  % support changing the page widths on demand
%\usepackage{tabularx}  % an enhanced tabular environment

%% enumitem: support pausing and resuming enumerate environments.
%\usepackage{enumitem}

%% rotating: provides two environments, sidewaystable and sidewaysfigure,
%% for typesetting tables and figures in landscape mode.  
%\usepackage{rotating}

%% subfig: provides for including subfigures within a figure,
%% and includes being able to separately reference the subfigures.
\usepackage{subfig}


\usepackage[bookmarks,bookmarksnumbered,%
    citebordercolor={0.8 0.8 0.8},filebordercolor={0.8 0.8 0.8},%
    linkbordercolor={0.8 0.8 0.8},%pagebordercolor={0.8 0.8 0.8},
    urlbordercolor={0.8 0.8 0.8},
    linktocpage%
    ]{hyperref}
%%\usepackage[bookmarks,bookmarksnumbered,%
%%    citebordercolor={0.8 0.8 0.8},filebordercolor={0.8 0.8 0.8},%
%%    linkbordercolor={0.8 0.8 0.8},pagebordercolor={0.8 0.8 0.8},%
%%    urlbordercolor={0.8 0.8 0.8},%
%%    pagebackref,linktocpage%
%%    ]{hyperref}
%% The following change how the the back-references text is typeset in a
%% bibliography when `backref' or `pagebackref' are used
%\renewcommand\backrefpagesname{\(\rightarrow\) pages}
%\renewcommand\backref{\textcolor{greytext} \backrefpagesname\ }

%allow use of \degree command
\usepackage{gensymb}

\usepackage{lineno}
\linenumbers



%% The following is used for tables of equations.
\newcounter{saveEq} 
\def\putEq{\setcounter{saveEq}{\value{equation}}} 
\def\getEq{\setcounter{equation}{\value{saveEq}}} 
\def 
\tableEq{ 

% equations in tables
\putEq \setcounter{equation}{0} 
\renewcommand{\theequation}{T\arabic{table}.\arabic{equation}} \vspace{-5mm} } 
\def\normalEq{ 

% renew normal equations
\getEq 
\renewcommand{\theequation}{\arabic{section}.\arabic{equation}}}
\newcommand{\normal}[2]{\ensuremath{N(#1,#2)}}

%%%%%%%%%%%%%%%%%%%%%%%%%%%%%%%%%%%%%%%%%%%%%%%%%%%%%%%%%%%%%%%%%%%%%%
%%%%%%%%%%%%%%%%%%%%%%%%%%%%%%%%%%%%%%%%%%%%%%%%%%%%%%%%%%%%%%%%%%%%%%
%%
%% Some special settings that controls how text is typeset
%%
 \raggedbottom      % pages don't have to line up nicely on the last line
% \sloppy       % be a bit more relaxed in inter-word spacing
 \clubpenalty=10000 % try harder to avoid orphans
 \widowpenalty=20000    % try harder to avoid widows
% \tolerance=1000




\title{Stock Reduction Analysis using catch at length data}            
\author{Catarina Wor, Roberto Licandeo,  Brett van Poorten, Carl Walters}
%\pgfrealjobname{pgfSweave-vignette}
%pgfSweave-vignette
\begin{document}

\maketitle

\begin{abstract}
   Last thing to be written
\end{abstract}


\section{Introduction}

%(old intro written by Carl Walters)
Modern stock assessments typically attempt to fit population dynamics models to catch at age and catch at length data, in hopes of extracting information from these data about age/size vulnerability and fishing mortality patterns \citep{methot_stock_2013,hilborn_quantitative_1992}.  In cases where age data are lacking, models like MULTIFAN-CL attempt to obtain estimates of vulnerability and fishing mortality only from size distribution data \citep{fournier_multifan-cl:_1998}. Combined with a few assumptions regarding the structure and variability in length at age, this procedure can even be used to attempt to recover information about changes in body growth patterns if there is a strong age-class signal in the length frequency data \citep{fournier_multifan-cl:_1998}.  Some assessment methods attempt to put aside the length frequency data, by converting these data to age compositions using age-from-length tables, perhaps using iterative methods to estimate proportions of fish at age for each length interval \citep{kimura_mixtures_1987}.  It is typical for assessment results from length-based assessment models to show substantial deviations between predicted and observed length distributions of catches, reflecting both sampling variation in the length composition data and incorrect assumptions about stability of growth and vulnerability patterns \citep{hilborn_quantitative_1992}. 
 
The vulnerability process is the combination of two processes: selectivity of the fishing gear and availability of the fished population in the area being fished \citep{beverton_dynamics_1957}. Both processes can vary over time and therefore modify the resulting selectivity. Although selectivity process can often be directly measured through gear experiments, availability is generally harder to measure as it depends on the distribution of the exploited population. Fish movement, changes in fish distribution, and changes in fleet distribution, can all affect availability and consequently lead to vulnerability changes. Changes in vulnerability are not uncommon \citep{sampson_exploration_2012} but are usually difficult to track over time. This difficulty is associated with an inability to distinguish between changes in fishing mortality and changes in vulnerability in most age and length based stock assessment methods. For this reason, many assessment methods rely on ad hoc parametric vulnerability models that may or may not include changes over time \citep{maunder_selectivity:_2014}. If misspecified, such models might lead to severe bias in fishing mortality estimates, which could result in misleading management advice \citep{martell_towards_2014}. 

 Here we suggest an alternative approach to assessment modeling that begins by assuming that the assessment model should exactly reproduce the observed catch at length distribution. This approach follows the dynamics of an age structred stock reduction analysis (SRA) \citep{walters_stochastic_2006} which follows a ``conditioned on catch'' format, subtracting observed catches at age from modeled numbers at age in estimation of numbers at age over time. This assumption is analogous to the classical assumption in virtual population analysis that reconstructed numbers at age should exactly match observed catch at age data \citep{hilborn_quantitative_1992}. The suggested approach may have two key advantages over statistical catch at age and/or catch at length models: (1) it does not require estimation of age or size vulnerability schedules, and (2) catch at length data are commonly available for every year, even when age composition sampling has not been conducted. 

 We this approach a Length-SRA assessment model. Here we present the model formulation, the demonstrate its performance with a simulation-evaluation analysis and apply it to real fisheries data from the Peruvian jack mackerel (\emph{Trachurus murphyi}) and Pacific Hake (\emph{Merluccius productus}) fisheries. 



\section{Methods}

In this section we describe the stock reduction analysis with catch at length data (Length-SRA), describe the simulation analysis and scenarios used to test the model and provide a description of the real data used to illustrate the model applicability.


\subsection{Stock reduction analysis with catch at length data - length-SRA}

The stock reduction analysis (SRA) described here proceeds through the following steps: (1) Compute numbers at age (based on recruitment estimates and mortality in the previous year); (2) Convert numbers at age into numbers at length based on the proportions of individuals at length given each age class; (3) Calculate the exploitation rate at length based on numbers at length and observed catch at length; (3) Convert the exploitation rate at length to exploitation rate at age; (4) Compute numbers in the following year using the exploitation rate at age, natural mortality and recruitment estimates.  


A crucial component of the length-SRA is the calculation of the proportions of individual at length given each age class ($P_{l|a}$ - eqs. \ref{Pla1.1}-\ref{Pla1.5}). The calculation of such proportions (eq. \ref{Pla1.1}) relies on four main assumptions regarding the distribution of length at age: (1) The mean length at age follows a von Bertalanffy growth curve (eq.\ref{Pla1.4}), (2) The length at age is normally distributed (eqs. \ref{Pla1.1} -\ref{Pla1.3}), (3) The standard deviation of the length at age is defined (e.g. eq.\ref{Pla1.5}) and (4) $P_{L|a}$ is constant for all lengths equal or greater than $L$ (eq.\ref{Pla1.3}).



 The proportions of length at age are used to convert the length-based quantities into age based quantities which are used to propagate the age structured population dynamics forward (Table \ref{tab:eqsall} - Population dynamics). We assume that recruitment follows a Beverton-Holt type recruitment curve (eq. \ref{PD2.1}), that harvesting occurs over a short, discrete season in each time step (year or shorter time period) and that natural survival rate is known and stable over time (eqs. \ref{PD2.1}-\ref{PD2.5}). We used the same model structure to simulate data and as an assessment model. The computation of numbers at age in the initial year (i.e first year in which data is reported) is different from that in the remaining years (eq. \ref{AO.1}). When the model is used as an stock assessment model, the recruitment in the initial year is given by the parameter $R_{init}$. This feature frees the model from the assumption that the population is at unfished equilibrium at the beginning of the data recording time series. 

 When the same model dynamics was used as an operating model, we initialized the model at unfished conditions (eq. \ref{OM4}) but only started  reporting data for the simulation evaluation procedure after the $init$ year. The period between the first time step and init functions is used as a burn-in period. On both the simulation and assessment models we used equilibrium yield per recruit quantities to calculate $MSY$ and $U_{MSY}$ (eqs.\ref{lz} to \ref{umsy}). These quantites depend on the vulnerability curves calculated for each year (eq. \ref{phieq}).

{}

\begin{table}
  %\centering
  %\large
\caption{ \large Indexes, variable definition and values used in simulation-evaluation}\label{tab:pop_dyn} 
\begin{tabular}{l c p{7.5cm}} 
\hline
Symbol & Value  & Description \\  
\hline
  $l$ & $\{l_o,...,L\}$ &Central point of length bin, $L=15$\\
  $a$ & $\{a_o,...,A\}$ &Age-class, $A=10$\\  
  $t$ & $\{1,...,T\}$ &Annual time step, $T=50$\\
  $a_o$ & $1$ &First age or age of recruitment \\
  $l_o$ & $1$ &Central point of first length bin \\
  $init$ & $21$ &Annual time step in which data starts to be reported \\
\hline
Distribution of length given age&&\\
${L_{inf}}$ &10 & Maximum average length\\
${k}$ &0.3& Rate of approach to $L_{inf}$\\
${t_o}$& -0.1&Theoretical time in which length of individuals is zero\\
$cv_l$ & 0.08 & Coefficient of variation for length at age curve\\
$P_{l|a}$ & &Matrix of proportions of length at age\\
$\Phi$ && Standard normal distribution\\
$z1_{a,l}$ &&Normalized $z$ score for lower limit length bins\\
$z2_{a,l}$&& Normalized $z$ score for upper limit length bins \\
$b1_{l}$ &&  Lower limit of length bins\\
$b2_{l}$ &&  Upper limit of length bins \\    
$\bar{L}_a$&&   Mean length at age\\
$\sigma_{L_a}$ && Standard deviation of length at age\\
\hline
Population dynamics&&\\
$R_o$& 100& Average unfished recruitment\\
$\kappa$ &10&  Goodyear recruitment compensation ratio\\
$S$ &0.7& Natural survival \\
$\sigma_{rec} $& 0.4 & standard deviation for recruitment deviations\\
$w_t$& $\mathcal{N}(0,\sigma_{R})$ &  Recruitment deviations for  years \{init-A+1,...,T\} \\
$N_{a,t}$ &&  Numbers of fish at age and time\\
$SB_{t}$ && Spawning biomass at time\\
$mat_a$ && Proportion of mature individuals at age\\
$a_{rec}, b_{rec}$ && Beverton \& Holt stock recruitment parameters \\
$R_{init}$& & Recruitment in the year data starts to be reported \\
$U_{a,init}$ &&  Exploitation rate at age before data starts being reported. \\
$U_{a,t}$ && Exploitation rate at age and time\\
$U_{l,t}$ && Exploitation rate at length and time \\
$C_{l,t}$ && Catch at length and time\\
$N_{l,t}$ && Numbers at length and time \\
$lx_a$ && Unfished survivorship at age\\
$\Phi_e$ && Unfished average spawning biomass per recruit \\
\hline
\hline
\end{tabular}
\end{table}

\begin{table}
  %\centering
  %\large
\caption{ \large Indexes, variable definition for operating model and MSY quantities}\label{tab:pop_dyn} 
\begin{tabular}{l c p{7.5cm}} 
\hline
Symbol & Value  & Description \\  
\hline
Operating model &&\\
$sel_{l,t}$ && Fishing selectivity at length and time\\
$g,a,b$ && Parameters for selectivity function \\
$U_{t}$ &&  Annual maximum exploitation rate \\
$C_{l,t}$ && Catch at length and time\\
$N_{l,t}$ && Numbers at length and time\\
$I_{t}$ && Index of abundance at time \\
$VB_{t}$ && Biomass that is vulnerable to the survey at time \\
$q$ &1.0& catchability coefficient \\
\hline
MSY quantities &&\\
$lz_a$&&\\
$U_z$&seq(0.0,1.0,by=0.001)& Hypothetical exploitation rates to calculate $MSY$\\
$\Phi_z$ &&Unfished average spawning biomass per recruit\\
$\Phi_{eq}$&&Fished under $U_z$ average spawning biomass per recruit\\
$sel_a$&& Selectivity at age \\
$R_{eq}$ &&Average fished recruitment under $U_z$\\
$Yield_z$ &&Equilibrium yield under $U_z$\\
$MSY$  && Maximum sustainable yield based on optimum spawner per recruit\\   
$U_{MSY}$ &&Exploitation rate that leads to maximum sustainable yield \\  
\hline
\hline
\end{tabular}
\end{table}


\begin{table}
  %\centering
  %\large
\caption{population dynamics for Length-SRA and operating model}\label{tab:eqsall} 
\tableEq
    \begin{align}
          % \hline
        %\hline \nonumber \\ 
            \hline  
           \text{Distribution of length given age}&&\nonumber\\
           \hline
            \label{Pla1.1}
            P_{l|a} &= \int_{z1_{a,l}}^{z2_{a,l}}{ \Phi(z) dz} \\
            \label{Pla1.2}
            z1_{a,l} &= \frac{b1_{l} - \bar{L}_{a}}{\sigma_{L_{a}}} \\
            \label{Pla1.3}
            z2_{a,l} &= \begin{cases}\frac{b2_{l} - \bar{L}_{a}}{\sigma_{L_{a}}} &\quad l<L  \\
             1.0 &\quad l=L \\
             \end{cases}\\
            \label{Pla1.4} 
            \bar{L}_{a} &= L_{inf} \cdot (1 - \exp^{(-k\cdot(a-t_o))})  \\
            \label{Pla1.5} 
            \sigma_{L_{a}} &= \bar{L}_{a} \cdot cv_l  \\
            %\nonumber \\
            \hline  
            \mbox{Population dynamics}&&\nonumber\\
            \hline
            \label{PD2.1}
             N_{a,t>init}&=\begin{cases} \frac{a_{rec} \cdot SB_{t-1}}{1+ b_{rec} \cdot SB_{t-1}} \cdot e^{w_t}, &\quad a=a_o  \\
            %
            N_{a-1,t-1}\cdot S \cdot (1-U_{a-1,t-1}),&\quad a_o<a<A  \\
            %
            \frac{N_{a-1,t-1} \cdot S \cdot (1-U_{a-1,t-1})}{1- S \cdot (1-U{a,t})},&\quad a=A \\
            \end{cases}\\
            \label{PD2.2}
            U_{a,t}&=\sum_a{(P_{l|a}\cdot U_{l,t})}\\
            \label{PD2.3}
            U_{l,t}&=\frac{C_{l,t}}{N_{l,t}}\\
            \label{PD2.4}
            N_{l,t}&=\sum_a{(P_{l|a}\cdot N_{a,t})}\\
            \label{PD2.5}
            SB_{t}&= \sum_a(mat_a \cdot w_a \cdot N_{a,t}) \\ 
           %\nonumber \\
            \hline 
            \mbox{Initial year and incidence functions} &&\nonumber \\ 
            \hline 
            \label{AO.1}
            %N_{a=1,init}&= R_{init} \cdot e^{w_{init}} \\
             N_{a,init}&=\begin{cases} R_{init} \cdot e^{w_{init}} &\quad a=a_o  \\
            %
            N_{a-1,init}\cdot S \cdot(1-U_{a-1,init}) \cdot e^{w_{init-a+1}},&\quad a_o<a<A  \\
            %
            \frac{N_{a-1,init} \cdot S \cdot (1-U_{a-1,init}) }{1- S \cdot (1-U{a,init})} \cdot e^{w_{init-a+1}},&\quad a=A \\
            \end{cases}\\
            \label{AO.2}
            a_{rec}&=  \frac{\kappa}{\phi_e} \\
            \label{AO.4}
            b_{rec}&= \frac{\kappa-1}{R_o \cdot \phi_e}  \\
             \label{AO.5}
            \Phi_e &= \sum_a{lx_a \cdot mat_a \cdot w_a} \\
            \label{AO.6}
            lx_a &= \begin{cases} 1 , &\quad a=1 \\
            lx_{a-1} \cdot S ,&\quad 1<a<A \\ 
            \frac{lx_{a-1} \cdot S}{1-S} ,&\quad a=A \\
             \end{cases}\\       
             %\nonumber \\
             \hline  
            %\mbox{Operating model}&&\nonumber\\
            %\hline
            %\label{OM4}
            % N_{a,t=1}&= lx_a* R_o\\
            %\label{OM4.1}
            %U_{l,t}&= U_{t} \cdot sel_{l,t}\\ 
            %\label{OM4.2}          
            %C_{l,t}&= N_{l,t} \cdot  U_{l,t} \cdot P_{l|a}\\
            %\label{OM4.3}
            %sel_{l,t} &= \frac{1}{1-g}\cdot \left(\frac{1-g}{g}\right)^{g} \cdot \frac{e^{a\cdot g\cdot (b-l)}}{1+e^{a\cdot(b-l)}}\\
            %\label{OM4.4}
            %I_{t}&= q \cdot VB_{t} \cdot e^{(\mathcal{N}(0,\sigma_{I_t})}\\ 
        \hline \hline \nonumber
    \end{align}
    \normalEq
\end{table}


The stock assessment model estimates three main parameters: average unexploited recruitment $R_{0}$, recruitment compensation ratio $\kappa$ and recruitment in the initial year $R_{init}$ (we assume that the  data collection for the fisheries does not start when the population is at an unfished equilibrium state). In addition, the recruitment deviations $w_t$ are estimated for all cohorts observed in the model, that is, the number of recruitment deviations is equal to the number of years in the time series plus the number of age classes greater that recruitment age.  



\begin{table}
  %\centering
  %\large
\caption{MSY quantities and operating model}\label{tab:eqsall} 
\tableEq
    \begin{align}
           \hline
        %\hline \nonumber \\ 
            \mbox{Operating model}&&\nonumber\\
            \hline
            \label{OM4}
             N_{a,t=1}&= lx_a* R_o\\
            \label{OM4.1}
            U_{l,t}&= U_{t} \cdot sel_{l,t}\\
            \label{OM4.2}          
            C_{l,t}&= N_{l,t} \cdot  U_{l,t} \cdot P_{l|a}\\
            \label{OM4.3}
            sel_{l,t} &= \frac{1}{1-g}\cdot \left(\frac{1-g}{g}\right)^{g} \cdot \frac{e^{a\cdot g\cdot (b-l)}}{1+e^{a\cdot(b-l)}}\\
            \label{OM4.4}
            I_{t}&= q \cdot VB_{t} \cdot e^{(\mathcal{N}(0,\sigma_{I_t})}\\ 
            \hline  
           \text{MSY quantities}&&\nonumber\\
           \hline
            \label{lz}
            lz_a &= \begin{cases} lz_a=1 &\quad a=a_o\\
              lz_{a-1}\cdot S \cdot (1-U_z) &\quad a_o<a<A\\
              \frac{lz_{a-1}\cdot S \cdot (1-U_z)}{1-S*(1-U_z)}&\quad a=A\\\end{cases}\\
            \label{phiz}
            \Phi_z &= \sum_a{lz_a \cdot mat_a \cdot w_a}\\
            \label{phieq}
            \Phi_{eq} &= \sum_a{lz_a \cdot sel_a \cdot w_a} \\
             \label{Req}
             R_{eq} &= R_o \cdot \frac{\kappa - \Phi_e/\Phi_z}{\kappa-1}\\
             \label{yz}
             Yield_z&=U_z*R_{eq}*\Phi_{eq}\\
             \label{msy}
             MSY&=max(Yield_z)\\
             \label{umsy}
             U_{MSY} &= U_z \rightarrow  max(Yield_z)\\
        \hline \hline \nonumber
    \end{align}
    \normalEq
\end{table}



The objective function (eq. \ref{Obj}) is composed of a negative log likelihood component, three penalties and a prior component for the recruitment compensation ratio $\kappa$. The negative log likelihood component minimizes the differences between the predicted and observed index of abundance (eq. \ref{PLL.1}). We assume that such differences are lognormally distributed (eqs. \ref{PLL.3}-\ref{PLL.4}) and use the conditional maximum likelihood estimator described by \citet{walters_calculation_1994} to estimate the catchability coefficient $q$ (eq. \ref{PLL.2}). A lognormal penalty is added to the negative log-likelihood function to constrain annual recruitment residuals so that the estimates have mean of zero and fixed standard deviation $\sigma_{R}$ (eq. \ref{pen.1}). the second  penalty, $P_{U_{max}}$ was implemented in order to prevent estimated values of exploitation rate at length greater than one ($U_{l,t}>1$) (eq. \ref{pen.3}) and the third penalty term $P_{U}$ was added in order to limit variability in vulnerability at length across the years, and therefore limit the influence of observation error on vulnerability estimates. Lastly, a normal prior for $log(\kappa)$ was included in the objective function (eq. \ref{pr.1}).



%\begin{table}
  %\centering
  %\large
%\caption{Estimated parameters, likelihood functions and penalties}\label{tab:LLpen} 
%\begin{tabular}{l p{7.cm}} 
%\hline
%Likelihood &\\
%$Z_t = log(I_t)-log(VB_{t})$&\\\\
%$q = e^(\bar{Z})$ & Estimate of the catchability coefficient\\\\
%$Zstat_t = Z_t - \bar{Z}$&\\\\ 
%$\mathcal{L}_{I_{t}}= \sum(-log(\mathcal{N}(Zstat_t; 0,\sigma_{I_t})))$& %Negative log likelihood\\\\
%\hline
%Penalties&\\
%$posfun(1-U_{l,t},0.01,fpen)$ & The posfun function is a penalty function implemented in ADMB\\
%$pen(U)=\frac{\sum^{t}\sum^{l}{U_{l,t}}}{T}$ &Additional penalty to prevent values of $U_{l,t} > 1$ \\\\
%$pen(w_t)=\sum (- log(\mathcal{N}(w_t; 0,\sigma_{R})))$ & Penalty on recruitment residuals \\\\
%\hline
%Priors&\\
%$prior(log(\kappa)) = \mathcal{N}( log(10), 0.9)$&\\\\
%\hline
%Objective function&\\
%$Obj = \mathcal{L}_{I_{t}} + pen(w_t)+ pen(U) + prior(log(\kappa))$  &\\
%\hline
%\hline
%\end{tabular}
%\end{table}


\begin{table}
\caption{Likelihood functions and penalties}\label{tab:LLpen} 
\tableEq
    \begin{align}
          % \hline
        %\hline \nonumber \\ 
            \hline  
           \text{Conditional Likelihood }&&\nonumber\\
           \hline
            \label{PLL.1}
            &Z_t = log(I_t)-log(VB_{t})\\
            \label{PLL.2}
            &q = e^{\bar{Z}} \\
            \label{PLL.3}
            &Zstat = Z - \bar{Z}  \\
            \label{PLL.4} 
            &LL_{1} \sim \mathcal{N}(Zstat| \mu=0,\sigma=\sigma_{I_t})\\
            \mbox{Penalties}&&\nonumber\\
            \hline
            \label{pen.1}
            &P_{wt} \sim \mathcal{N}(wt|\mu=0,\sigma=\sigma_{R})\\
            %\label{pen.2}
            % posfun(1-U_{l,t},0.01,fpen) \\       
            &\mu_{U_{t-2:t}} = \left(\sum_{t-2}^{t}{\frac{U_{l,t}}{\bar{U_{t}}}}\right)/3\\
            &U^{pen}_{t}={\left(\frac{U_{l,t}}{\bar{U_{t}}} - \mu_{U_{t-2:t}}\right)}^{2} \\
            &SS_{vul}=\sum_{t=3}^{t=T}{U^{pen}_{t}}\\
            \label{pen.3}
            &P_{U}=\frac{SS_{vul}}{\sigma_{vul}}\\
            \label{pen.4}
            &P_{U_{max}}=\sum^{t}\sum^{l}{{U_{t,l}}^{10}}\\
            %penmaxUy(syr) = sum(pow( Ulength( syr )(1,nlen),10));
            \hline 
            \mbox{Priors} &&\nonumber \\ 
            \hline 
            \label{pr.1}
            &prior(log(\kappa)) \sim \mathcal{N}(log(true \kappa), 0.9)\\
             \hline  
            \mbox{Objective function}&&\nonumber\\
            \hline
            \label{Obj}
            &Obj = -log(LL_{1}) + -log(P_{wt}) + P_{U} + P_{U_{max}} + prior(log(\kappa))\\ 
        \hline \hline \nonumber
    \end{align}
    \normalEq
\end{table}



\subsection{Simulation evaluation}

In order to perform a simulation evaluation of the Length-SRA under various scenarios we used the same model dynamics described in Table \ref{tab:eqsall} - Population dynamics, as an operating model. However we modified the model to control annual exploitation rate (eq. \ref{OM4.1}), time varying selectivity (eq. \ref{OM4.3}), and observation and process errors. Selectivity in the operating model was computed with the three parameter selectivity equation described by \citet{thompson_confounding_1994} (eq. \ref{OM4.3}). The observation error in the operating model included lognormal error in the index of abundance and logistic multivariate error in the catch numbers at length. Recruitment deviations were assumed to be lognormally distributed.   


We considered a total of six different scenarios in the simulation evaluation runs. Three different historical exploitation rate trajectories were used: contrast, one way trip and $U$ ramp. In the contrast scenario the exploitation rate ($U_t$) starts low and increases beyond $U_{msy}$ and then decreases until $U_t=U_{msy}$. In the one way trip scenario $U$ increased through time until $U=2 \cdot U_{msy} $. In the $U$ ramp scenario, $U_t$ increases steadily until $U_t=U_{msy}$ and remains constant thereafter. In addition to the exploitation rate scenarios, we considered two selectivity scenarios: constant and time varying selectivity. In the constant selectivity scenario, selectivity was assumed to follow a sigmoid shape. In the time varying selectivity scenario, the selectivity curve was assumed to vary every year, progressively changing form a dome shaped curve to sigmoid and back to dome shaped. The complete list of scenarios and the acronym used for them is presented in Table \ref{tab:simeval}. 

All simulation runs had 30 years of data and we used 100 simulation runs for each scenario. We evaluated the distribution of the \% relative error ($ \frac{esimated-simulated}{simulated}$) for the main parameter estimates ($R_{0}$, $R_{init}$ and $\kappa$) and for four derived quantities (Depletion: $\frac{B_t}{B_{0}}$, MSY, $U_{MSY}$ and $q$). 


\begin{table}[ht!]
\centering
\caption{ \large Simulation-estimation scenarios}
\label{tab:simeval} 
 \begin{tabular}{c c c} 
 \hline
 Scenario Code & Selectivity  & $U$ trajectory \\ 
 \hline
 CC     & constant   &  contrast    \\ 
 CO     & constant   & one way trip  \\ 
 CR     & constant   & $U$ ramp  \\ 
 VC     & time-varying   &  contrast    \\ 
 VO     & time-varying   & one way trip  \\ 
 VR     & time-varying   & $U$ ramp  \\  
 \hline
\end{tabular}
\end{table}


\subsection{Real data examples}

Two species were chosen to illustrate the application of the Length-SRA to real datasets: Chilean jack mackerel and Pacific hake. Both species are believed to be subject to time varying selectivity. 

The Pacific hake fishery is believed to exhibit time varying selectivity due to cohort targeting and annual changes fleet spatial distribution. The population is know to have spasmodic recruitment, with high recruitment events occurring once or twice every decade \citep{ressler_pacific_2007}. Pacific hake tends to segregate by size during their annual migration\citep{ressler_pacific_2007}, allowing the fishing fleet to target the strong cohorts by changing the spatial distribution of fishing effort as the cohort ages. Hake catch at length data was available for the period between 1975 and 2013. The survey index of abundance was available itermitently from 1995 to 2013. 

The movement pattern of jack mackerel is not as well known, although fish appear to move between spawning and feeding areas \citep{gerlotto_insight_2012}. Variability in selectivity patterns for the jack mackerel fishery are believed to be associated both with evolution of fleet capacity and gear utilization and with compression and expansion of the species range associated with abundance changes \citep{gerlotto_insight_2012}. Jack mackerel catch at length data was available from 1980 to 2013 and the survey index was avaialble for the years between 1986 and 2013, with the exception of the year of 2010. 



\section{Results}

\subsection{Simulation-evaluation}


Even though the parameter estimates were not entirely unbiased, the amount of bias was relatively low, especially for the leading parameters $R_0$ and $\kappa$ (Figure \ref{mainParams}). The parameter  $R_0$ was slightly  accurately estimated for the contrast and one-way-trip scenarios, with median \% errors  between -1.8\% and 2.5\%. In the $U$ ramp scenarios, $R_0$ was underestimated with with median \% errors of -5.9\% (CR) and -10.6\% (VR). The parameter $R_{init}$ was underestimated for all scenarios with relatively high median \% errors varying between -14.4\% and -33.9\%. Once again the $U$ ramp scenarios produced the highest bias (Figure \ref{mainParams}). The $\kappa$ parameter was the most accurately and precisely estimated with median \% errors varying between 0.5\% and 2.4\%.



\begin{figure}[htbp]
  \centering
  \includegraphics[scale=.65]{/Users/catarinawor/Documents/Length_SRA/R/plots/figs/main_params_publ.pdf}
  \caption{Relative error (\%) for main parameters for all scenarios considered in the simulation-evaluation. }
\label{mainParams}
\end{figure}



In relation to the derived quantities (Figure \ref{derivQuant}), the Length-SRA model tended to underestimate depletion with median \% errors ranging between -17.0\% and -1.3\%.  $MSY$ was also underestimated for all scenarios, with median \% errors ranging between - 15.8\% and -4.6\%. The estimates for $U_{MSY}$ showed very low ($<$3\%) median \% errors for all scenarios, and were underestimated for the time varying selectivity scenarios and overestimated for the constant selectivity scenarios. The estimates of $q$ were positively biased, particularly for the the $U$ ramp scenarios (CR and VR), this is likely to be associated with the underestimations of both $R_0$ and $R_{init}$ for those scenarios.

\begin{figure}[htbp]
  \centering
  \includegraphics[scale=.65]{/Users/catarinawor/Documents/Length_SRA/R/plots/figs/derivQuant_publ.pdf}
  \caption{Relative error (\%) for main parameters for all scenarios considered in the simulation-evaluation. }
\label{derivQuant}
\end{figure}

The simulation-evaluation exercise showed that the Length-SRA model is able to track selectivity changes over time (Figure \ref{sel}). However, the estimates of selectivity are less accurate in the initial year of data (year 21) for all scenarios and particulatly unnacurate for the Uramp scenarios. 
Precision and accuracy of selectivity estimates were better for the contrast and one-way trip scenarios for both the constant and time-varying selectivity. 


\begin{figure}[htbp]
  \centering
  \includegraphics[scale=.7]{/Users/catarinawor/Documents/Length_SRA/R/plots/figs/sel_publ.pdf}
  \caption{Simulated and realized selectivity estimates for a set of years within simulation-evaluation time series.}
\label{sel}
\end{figure}



\subsection{Real data examples}


The model fit to the Pacific hake and jack mackerel indexes of abundance relatively well (Figure \ref{it}). The model fit for both species resulted in time varying selectivities that lead to variability in the $MSY$ and $U_{MSY}$ estimates. 




\begin{figure}[htbp]
  \centering
  \includegraphics[scale=.85,width=\textwidth]{/Users/catarinawor/Documents/Length_SRA/R/plots/figs/real_examples.pdf}
  \caption{Fit to index of abundance for Pacific hake and jack mackerel.}
\label{it}
\end{figure}






\section{Discussion}







We present a length-based stock reduction analysis (SRA) that allows monitoring of time varying selectivity.  In this model catch at length is assumed to be known without error and exploitation rate at length is derived directly from the estimates of numbers at length. This fact is important because it allows the mode to bypass the requirement for the estimation of a selectivity ogive, as is required in other length based models \citep{sullivan_catch-at-length_1990,fournier_multifan-cl:_1998}.  

Selectivity parameters can be particularly hard to estimate, especially if it changes over space or time \citep{martell_towards_2014}. Accurate estimates of selectivity are particularly important if the fishery is managed based on yield per recruit reference points. The yield per recruit of a fishery depends on the selectivity curve \citep{vasilakopoulos_unfulfilled_2016} and for this reason, changes  in selectivity over time will directly affect values reference points. 



The approach used in the length-based SRA described here is analogous to that used in virtual population analysis (VPA). Here we assume that catch at length in numbers is assumed to known without error. Therefore selectivity estimates are the result of the estimated exploitation rate and observation and sampling error. Because of this assumption, it is important that the catch at length sampling is representative of the total removals from the population. Biased sampling will lead to biased estimates of selectivity and result in bias in the fishery reference points.  


  
Main talking points:
\begin{itemize}
  \item The model is not that good, main parameters are biased and selectivity estimates are not particularly accurate or precise
  \item Why is this failing?
  \item Is this still useful?
  \item Depletion and MSY are underestimated -  the model tends to produce conservative benchmarks.
  \item U ramp scenario yield very bad results - lack of information in the time series.
  \item  Time- varying growth might result in even worse results. Suggest estimating  cohort-specific growth curves or implementing the density dependent functions shown in multifan-CL?
  \item Further testing of this model in a closed-loop simulation set up would provide more insight on the model performance on achieving management outcomes
\end{itemize}






 



\section{Acknowledgments}
 We would like to thank Allan Hicks and NOAA for the provision of Pacific hake data and ????? for the provision of Jack mackerel data. 

\clearpage

\bibliographystyle{apa}
\bibliography{references}


\end{document}
