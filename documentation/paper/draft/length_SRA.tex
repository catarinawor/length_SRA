% stuff to be used whe compiling only one chapter -- e.e for revision and committee approval

\documentclass{article}

%% Compile this with
%% R> library(pgfSweave)
%% R> pgfSweave

%\usepackage[nogin]{Sweave} %way to understand R code
\usepackage{pgf} %look it up
\usepackage{tikz} % allows R code or other stuff in the figure labels
\usepackage{rotating}
\usepackage{color}
\definecolor{greytext}{gray}{0.5}
\usepackage[left=1.2in,right=1.2in,top=1.2in,bottom=1.2in]{geometry} % sets the margins
\usepackage{fancyhdr}
\usepackage{setspace}
\usepackage{indentfirst}
\usepackage{titlesec}
\usepackage{natbib}
\usepackage{checkend}   % better error messages on left-open environments
\usepackage{graphicx}   % for incorporating external images

\newcommand{\lang}{\textsf} %
\newcommand{\code}{\texttt}
\newcommand{\pkg}{\texttt}
\newcommand{\ques}[1]{{\bf\large#1}}
\newcommand{\eb}{\\ \nonumber} 

\doublespacing

\usepackage{times,mathptmx,courier}
\usepackage[scaled=1]{helvet}
\usepackage{lineno}
\usepackage{listings}
\lstset{basicstyle=\sffamily\scriptsize,showstringspaces=false,fontadjust}

\usepackage{fixme}%you probably wan't this 
%% Math or theory people may want to include the handy AMS macros
\usepackage{amssymb}
\usepackage{amsmath}
\usepackage{amsfonts}
\usepackage{float}

%% The pifont package provides access to the elements in the dingbat font.   
%% Use \ding{##} for a particular dingbat (see p7 of psnfss2e.pdf)
%%   Useful:
%%     51,52 different forms of a checkmark
%%     54,55,56 different forms of a cross (saltyre)
%%     172-181 are 1-10 in open circle (serif)
%%     182-191 are 1-10 black circle (serif)
%%     192-201 are 1-10 in open circle (sans serif)
%%     202-211 are 1-10 in black circle (sans serif)
%% \begin{dinglist}{##}\item... or dingautolist (which auto-increments)
%% to create a bullet list with the provided character.
\usepackage{pifont}
\usepackage{fixltx2e}
\usepackage[printonlyused,nohyperlinks]{acronym}

\usepackage{comment}



\titleformat*{\section}{\singlespacing\raggedright\bfseries\Large}
\titleformat*{\subsection}{\singlespacing\raggedright\bfseries\large}
\titleformat*{\subsubsection}{\singlespacing\raggedright\bfseries}
\titleformat*{\paragraph}{\singlespacing\raggedright\itshape}

%% The caption package provides support for varying how table and
%% figure captions are typeset.
\usepackage[format=hang,indention=-1cm,labelfont={bf},margin=1em]{caption}

%% url: for typesetting URLs and smart(er) hyphenation.
%% \url{http://...} 
\usepackage{url}
\urlstyle{sf}   % typeset urls in sans-serif
%\graphicspath{{/home/daft/Dropbox/figures/}{/home/daft/Dropbox/figures/cpue_figures/}} % set the path to figures folder
%%%%%%%%%%%%%%%%%%%%%%%%%%%%%%%%%%%%%%%%%%%%%%%%%%%%%%%%%%%%%%%%%%%%%%
%%%%%%%%%%%%%%%%%%%%%%%%%%%%%%%%%%%%%%%%%%%%%%%%%%%%%%%%%%%%%%%%%%%%%%
%%
%% Possibly useful packages: you may need to explicitly install
%% these from CTAN if they aren't part of your distribution;
%% teTeX seems to ship with a smaller base than MikTeX and MacTeX.
%%
%\usepackage{pdfpages}  % insert pages from other PDF files
%\usepackage{longtable} % provide tables spanning multiple pages
%\usepackage{chngpage}  % support changing the page widths on demand
%\usepackage{tabularx}  % an enhanced tabular environment

%% enumitem: support pausing and resuming enumerate environments.
%\usepackage{enumitem}

%% rotating: provides two environments, sidewaystable and sidewaysfigure,
%% for typesetting tables and figures in landscape mode.  
%\usepackage{rotating}

%% subfig: provides for including subfigures within a figure,
%% and includes being able to separately reference the subfigures.
\usepackage{subfig}


\usepackage[bookmarks,bookmarksnumbered,%
    citebordercolor={0.8 0.8 0.8},filebordercolor={0.8 0.8 0.8},%
    linkbordercolor={0.8 0.8 0.8},%pagebordercolor={0.8 0.8 0.8},
    urlbordercolor={0.8 0.8 0.8},
    linktocpage%
    ]{hyperref}
%%\usepackage[bookmarks,bookmarksnumbered,%
%%    citebordercolor={0.8 0.8 0.8},filebordercolor={0.8 0.8 0.8},%
%%    linkbordercolor={0.8 0.8 0.8},pagebordercolor={0.8 0.8 0.8},%
%%    urlbordercolor={0.8 0.8 0.8},%
%%    pagebackref,linktocpage%
%%    ]{hyperref}
%% The following change how the the back-references text is typeset in a
%% bibliography when `backref' or `pagebackref' are used
%\renewcommand\backrefpagesname{\(\rightarrow\) pages}
%\renewcommand\backref{\textcolor{greytext} \backrefpagesname\ }

%allow use of \degree command
\usepackage{gensymb}

\usepackage{lineno}
\linenumbers



%% The following is used for tables of equations.
\newcounter{saveEq} 
\def\putEq{\setcounter{saveEq}{\value{equation}}} 
\def\getEq{\setcounter{equation}{\value{saveEq}}} 
\def 
\tableEq{ 

% equations in tables
\putEq \setcounter{equation}{0} 
\renewcommand{\theequation}{T\arabic{table}.\arabic{equation}} \vspace{-5mm} } 
\def\normalEq{ 

% renew normal equations
\getEq 
\renewcommand{\theequation}{\arabic{section}.\arabic{equation}}}
\newcommand{\normal}[2]{\ensuremath{N(#1,#2)}}

%%%%%%%%%%%%%%%%%%%%%%%%%%%%%%%%%%%%%%%%%%%%%%%%%%%%%%%%%%%%%%%%%%%%%%
%%%%%%%%%%%%%%%%%%%%%%%%%%%%%%%%%%%%%%%%%%%%%%%%%%%%%%%%%%%%%%%%%%%%%%
%%
%% Some special settings that controls how text is typeset
%%
 \raggedbottom      % pages don't have to line up nicely on the last line
% \sloppy       % be a bit more relaxed in inter-word spacing
 \clubpenalty=10000 % try harder to avoid orphans
 \widowpenalty=20000    % try harder to avoid widows
% \tolerance=1000




\title{Stock Reduction Analysis using catch at length data}            
\author{Catarina Wor, Brett van Poorten, Roberto Licandeo, Carl Walters}
%\pgfrealjobname{pgfSweave-vignette}
%pgfSweave-vignette
\begin{document}

\maketitle

\begin{abstract}
   Last thing to be written
\end{abstract}


\section{Introduction}

%(old intro written by Carl Walters)
Modern stock assessments typically attempt to fit population dynamics models to catch at age and catch at length data, in hopes of extracting information from these data about age/size vulnerability and fishing mortality patterns \citep{methot_stock_2013,hilborn_quantitative_1992}.  In cases where age data are lacking, softwares like MULTIFAN-CL attempt to obtain estimates of vulnerability and fishing mortality only from size distribution data \citep{fournier_multifan-cl:_1998}. Combined with a few assumptions regarding the structure and variability in length at age, this procedure can even be used to attempt to recover information about changes in body growth patterns if there is a strong age-class signal in the length frequency data \citep{fournier_multifan-cl:_1998}.  Some older assessment methods attempt to put aside the length frequency data, by converting these data to age compositions using age-from-length tables, perhaps using iterative methods to estimate proportions of fish at age for each length interval (e.g. Shirippa and Goodyear, ref.).  It is typical for assessment results from length-based assessment models to show substantial deviations between predicted and observed length distributions of catches, reflecting both sampling variation in the length composition data and incorrect assumptions about stability of both growth and vulnerability patterns. 
 

 Here we suggest an alternative approach to assessment modeling that begins by assuming that the assessment model should exactly reproduce the observed catch at length distribution.  This is similar to the classical assumption in virtual population analysis that reconstructed numbers at age should exactly match observed catch at age data, or the suggestion by Schnute (ref) that statistical catch at age models might best be run in a “conditioned on catch” format by subtracting observed catches at age from modeled numbers at age in estimation of numbers at age over time.  The suggested approach may have two key advantages over statistical catch at age and/or catch at length models: (1) it does not require estimation of age or size vulnerability schedules, and (2) catch at length data are commonly available for every year, even when age composition sampling has not been conducted.  




\section{Methods}

In this section we describe the stock reduction analysis with catch at length data, describe the simulation analysis and scenarios used to test the model and provide a description of the real data used to illustrate the model applicability.




\subsection{Stock reduction analysis with catch at length data}

The stock reduction analysis described here starts by calculating , the proportions of individual at length for each age class (Table \ref{tab:length_age}. The calculation of such proportions relies on three main assumptions regarding the distribution of length at age: (1) The mean length at age follows a von Bertalanffy growth curve \ref{T1.4}, (2) The length at age is normally distributed \ref{T1.1} and (3) The standard deviations of the length at age distributions is given by the product of the mean length at age and a constant cv \ref{T1.5}. 

	
\begin{table}
  %\centering
  \Large
\caption{ \large Age at Length}\label{tab:length_age} 
\tableEq
    \begin{align}
           \hline
       		&\mbox{Variable definition} \nonumber \\ 
            &\mbox{$P_{l|a}$ = Matrix of proportions of lengh  at age} &\nonumber\\ 
            &\mbox{$z1_{a,l}$ = Normalized Z score for lower limit length bins}&\nonumber\\ 
            &\mbox{$z2_{a,l}$ = Normalized Z score for upper limit length bins}&\nonumber\\
            &\mbox{$b1_{l}$ = Lower limit of length bins}&\nonumber\\ 
            &\mbox{$b2_{l}$ = upper limit of length bins}&\nonumber\\ 
            &\mbox{$\bar{L}_a$ = Mean length at age}&\nonumber\\ 
            &\mbox{$\sigma_{L_a}$ = Standard deviation of lengh at age}&\nonumber\\ 
            &\mbox{${L_{inf}}$ =  Maximum average length }&\nonumber\\ 
            &\mbox{${k}$ =  rate of approach to $L_{inf}$}&\nonumber\\
            &\mbox{${t_o}$ =  Theoretical time in which length of individuals is zero}&\nonumber\\ 
            &\mbox{$cvL$ = Coefficient of variation for length curve}&\\
        \hline \nonumber \\
        &\mbox{Age-schedule information} \nonumber \\ \nonumber \\
            %
            P_{l|a}&= \int_{z1_{a,l}}^{z2_{a,l}}{ \mathcal{N}(0,1)}  \label{T1.1}\\
            z1_{a,l} &= \frac{b1_{l} - \bar{L}_{a}}{\sigma_{L_{a}}} \label{T1.2}\\
            z2_{a,l} &= \frac{b2_{l} - \bar{L}_{a}}{\sigma_{L_{a}}} \label{T1.3}\\
            \bar{L}_{a} &= L_{inf} \cdot (1 - \exp^{(-k\cdot(a-t_o))})  \label{T1.4}\\
            \sigma_{L_{a}} &= \bar{L}_{a} \cdot cvL  \label{T1.5}\\
            \nonumber \\
        \hline \hline \nonumber
    \end{align}
    \normalEq
\end{table}

{}
Once the length composition at age has been determined it can be used to convert the length based quantities in the model into age based quantities.  , The next step is to propagate the population dynamics forward with a age structured population dynamics model. We assume that recruitment follows a Beverton and Holt type recruitment curve, that harvesting occurs over a short, discrete season in each time step (year or shorter time period) and that natural survival rate is stable over time (Table \ref{tab:pop_dyn})
	
\begin{table}
  %\centering
  \Large
\caption{ \large Age at Length}\label{tab:pop_dyn} 
\tableEq
    \begin{align}
           \hline
       		&\mbox{Variable definition} \nonumber \\ 
            &\mbox{$N_{a,t}$ = Numbers of fish at age and time} &\nonumber\\ 
           \hline \nonumber \\
        &\mbox{Age-schedule information} \nonumber \\ \nonumber \\
            %
            N_{a,t}&=\begin{cases} \frac{a \cdot SB_{t-1}}{1+ b \cdot SB_{t-1}} \cdot exp(wt), &\quad a=1 \\ \\
            %
            N_{a-1,t-1}\cdot S_{a}\cdot(1-U_{a-1,t-1}),&\quad 1<a<A \\ \\
            %
            \frac{N_{a-1,t-1} \cdot S_{a} \cdot (1-U_{a-1,t-1})}{1- S_{a} \cdot 1-U{a,t}},&\quad a=A \\
            \end{cases}\label{T2.1}\\
            U_{a,t}=\sum_a{(P_{l|a}\cdot U_{l,t})}\label{T2.2}\\
            U_{l,t}=\frac{C_{l,t}}{N_{l,t}}\label{T2.3}\\
            N_{l,t}=\sum_a{(P_{l|a}\cdot N_{a,t})}\label{T2.3}\\
            \nonumber \\
        \hline \hline \nonumber
    \end{align}
    \normalEq
\end{table}

The model is then fit to data, this particular model includes 5 likelihoods components.  
\subsection{Simulation routines and scenarios}

\subsection{Real data examples}

equations and data description go here


\section{Results}

show figures with real and simulated data

\section{Discussion}

Does the model work

\clearpage

 \bibliographystyle{apa}
       \bibliography{references.bib}


\end{document}
